\label{section:examples}

Let us consider some relevant simple examples. 

\subsection{Piecewise linear function}

Consider the simple case where $f(x; \theta(t), t) = \theta(t) \in \R$ and $p=1$.
If $\theta(t) = \theta_0$ constant, then the solution $x(t)$ are simply linear functions. 
If our notion of smoothness is dictated by linear functions, that is, linear functions are consider smooth, we can then aim to find piecewise constant functions $\theta(t)$.

In the discrete case, we have the simple relationship $x_{i+1} = x_i + \theta_i \Delta t_i$, which is nothing else that the exact solution of the differential equation for this simple case. 
The discrete problem is then equivalent to solve 
\begin{equation}
    \min_{x_0 \, \theta}
    \| Y - X\theta - x_0 1_{n \times 1} \|_2^2
    + 
    \lambda
    \| D^{(1)}\theta \|_1
\end{equation}
with $x_0$ the initial condition (in principle, to be determined) and 
\begin{equation}
    X 
    = 
    \begin{bmatrix}
        \Delta t_1 & 0 &  &  \\
        \Delta t_1 & \Delta t_2 & 0 &  \\
        \vdots & \vdots & \ddots & 0 \\
        \Delta t_1 & \Delta t_2 & \dots & \Delta t_N
    \end{bmatrix}
    \qquad 
    D^{(1)}
    = 
    \begin{bmatrix}
        1 & -1 &  &  & 0 \\
         & 1 & -1 &  &  \\
         &  &  &  &  \\
        0 &  &  & 1 & -1
    \end{bmatrix}
\end{equation}
Now, here evaluating the loss funcion requires to solve the solution and the penalization is direclty been applied to the parameter. 
However, we can rewrite $\theta_i = (x_{i+1}-x_i)/\Delta t_i$ and instead apply the smooth penalization on the solution $x_i$ instead of $\theta$. 
We can then transform the problem to an equivalent synthesis problem \cite{synthesis-analysis}
\begin{equation}
    \min_x 
    \| Y - X \|_2^2
    + \lambda
    \| D^{(2)} X \|_1    .
\end{equation}
It is easy to see using the recursion of the matrices $D^{(k+1)} = D^{1} D^{(k)}$ that both problems are equivalent to each other \cite{tibs-trend-filter}. 

The difference between the analysis and synthesis approaches \cite{synthesis-analysis} in this context is equivalent to the difference between trajectory and gradient matching in dynamical data analysis \cite{ramsay2017dynamic}.

\subsection{Apparent polar wander path}

Paleomagnetism was one of the scientific disciplines that provided strong evidence in favor or the theory of plate tectonic motion during the past century. 
The idea is simple: when rocks are formed, they are magnetized in the same direction than the local magnetic field. 
If a new rock is formed in the present day (for example, when magma solidifies) close to the equator, it will acquire a remanent magnetization that will be close to be parallel to the surface, but if it is located close to the poles the magnetization will be perpendicular to the surface. 
Assuming that the magnetic field of the Earth is a dipole, measuring the remanent magnetization of rocks allows estimation of the relative position of the magnetic pole, which is define as \textit{paleomagnetic pole}. 
We can do this for rocks with different chronological dating and estimate the relative position of the magnetic north pole for each one of these. 
If well the locations of the magnetic pole migrate over time, in the time scale of continental drift the average magnetic pole coincides with the spin axis of the Earth. 
However, in reality we observe that the paleomagnetic poles move away gradually over time as we look to older rocks. 
This is caused by the movement of the rock itself and tracking how the paleomagnetic poles move allows geologists to reconstruct the history of a tectonic plate. 
This apparent movement of the paleomagnetic poles as we move backwards in time is called \textit{apparent polar wander path} and is the time series we are interested in exploring for this project. 

The initial paleomagnetic pole, that is, the expected paleomagnetic pole in the present, coincides with one of the geographical poles (by convention, here we will assume that is located in the geographic South pole). 
From the modelling perspective, this means that we do not need to worry about the initial condition of the system. 
Another important point is the level of uncertainty in the actual position of paleomagnetic poles. One single paleomagnetic pole is the result of multiple noisy measurement performed in same site. 
The final aggregate of all these measurements is reported as a new paleomagnetic pole with standard deviations in the order of $10^\circ$ or $20^\circ$. 
A cleaner view of the apparent polar wander path emerges when we average paleomagnetic poles over big temporal windows. 

Finally, the movement of plate tectonics can be described (to certain level of approximation) as a series of stable Euler rotations that persist during certain periods of time.
This is the case of a single tectonic plate moving with the same angular velocity around a certain Euler pole (different than the rotation axis of the Earth) for certain interval of time until some other dynamical process takes action, such as the collision of two plates, which results in a modification of the original trajectory of the first plate that can be described as a rotation but around a different axis and probably at different speed. 

Mathematically speaking, we can describe any path on the sphere with initial condition $x(t_0)=x_0$ as a three dimensional vector $x(t) \in \S2$ that is the solution of the differential equation 
\begin{equation}
    \frac{dx}{dt} = L(t) \times x(t),
    \quad 
    x(t_0) = x_0
\end{equation}
with $L(t) \in \R^3$ the angular momentum vector, that is, a vector with norm $\| L \|_2 = \omega$ equals to the angular velocity, and direction parallel to the axis rotation. 
Notice that for every choice of $L(t)$, the solution satisfies $\| x(t) \|_2 = 1$ for all times.

\subsection{General path on a manifold}

Given a manifold $\mathcal M \subset \R^3$ (this can be generalized to other dimensions), we can represent any possible continuous curve on $\mathcal M$ as the solution of the ordinary differential equation
\begin{equation}
    \frac{dx}{dt}
    = 
    L(t) \times n(x),
\end{equation} 
with $n(x)$ the normal vector to the manifold 